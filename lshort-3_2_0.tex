%double-click "\cmd" will show U help.pdf
%lshort CH1.4
\documentclass[a4paper,11pt]{book} %book,article,report,letter book(even,odd)...cool convenience
\usepackage{amsmath}
\usepackage{amssymb}
\usepackage{graphicx}
\usepackage{verbatim}
\usepackage[utf8]{inputenc}
    %\usepackage{xeCJK}
    %\setCJKmainfont{SimSun}
\author{William Song}
\title{Tex from Anywhere}

    %\DeclareUnicodeCharacter
%^以上为导言区域
\begin{document}
    %\maketitle title author datetime

\begin{comment} %from classes.pdf in D:\MiKTeX 2.9\doc\latex\base
                %from internet
\newcommand{新命令}[参数数量][默认值]{定义内容}
<*article|report|book>
\newcommand\@ptsize{}
\tableofcontents
\setlength\columnsep{10\p@}
\end{comment}

\mainmatter
    % in WinEdt, compile pdf order: close pdfViewer -> PDFLatex -> PDFPreview |->
    %                                                      ^------------------v
\part{chaos}
\chapter[lshort]{it's short really}% how insert ordinary bracket in latex-command's option's bracket[lshort[3.20]]
    %???\bigg\[ver3.20\bigg]
    %???{a \brack b}
\section{lshort Ch1}

\paragraph{lshort 1.3.1}white/invisible spaces

It does not matter whether you
enter one or several         spaces
after a word.

An empty line starts a new paragraph.

    %howto use chinese in latex command parameters or options like \chapter[特殊字符]{} or \paragraph{特殊字符}
\paragraph{lshort 1.3.2}specail charaters
\\  %需要注意的是,[()]不是特殊字符
\# \$ \% \^{O} \& \_ \{ \} \~{A} \LaTeXe
\\
    %\\[5pt]
$\backslash$
    %\\[-5pt]

\paragraph{lshort 1.3.3} \LaTeX{} command
\\  %tex 吞吃空间距
I read that Knuth divides the peopel working with \TeX{} into \TeX{}nicians and
\TeX perts.
\\
Today is \today.
\\  %比对  Martin Gardner PDF 出版物格式,了解英语国家排版设计基本惯例
\\  %紧接 section第一段文字起始点顶左界;之后表意分为\\(断行顶左界)和空行(新段落首行空3字符);以上均不同于 \paragraph{}(不带参数首行空2字符,带参数 参数段头顶左界)
\\  %笨办法实现样子上的分段且顶左界,编译警告“2 bad boxes”
Iou can \textsl{lean} on me!
    %\newline
    %???注释行不算做空行

Iou can \textsl{lean} on me!

\paragraph{lshort 1.3.4}comments
\\This is an % stupid
    % Better: instructive
example: Supercal%
                ifragilist%
        icexpialidocious
%usepackage{verbatim}
%!!!can NOT use in math envs
\begin{comment}
This is another
rather stupid,
but helpful
\end{comment}
example for embedding
comments in your document.

……

\paragraph{A}
Iou can \textsl{lean} on me!
\\%[-5pt] %
use a b c d e f g H I J K L m 0 1 2 3 4 5 6 7 8 9 10 a b c d e f
\linebreak[3] Emacs-LaTex inside
\\
use a b c d e f g H I J K L m 0 1 2 3 4 5 6 7 8 9 10 a b c d e f g H I J K L m
\\
% only undet book/reporte mode ...
balabal \ldots{} %...
\LaTeXe
\\
next
\\%\newline
time

\begin{comment}
so, so
you think
you can tell
heaven from hell
\end{comment}
%awlful... break pages

\subsection{nasa-giss}
Hypertext Help with LaTeX
Line and Page Breaking

The first thing LaTeX does when processing ordinary text is to translate your input file into a string of glyphs and spaces. To produce a printed document, this string must be broken into lines, and these lines must be broken into pages. In some environments, you do the line breaking yourself with the \\ command, but LaTeX usually does it for you.
Commands for line and page breaks
%    \hyphenation
%
%    \cleardoublepage
%    \clearpage
%
%    \newline
%    \linebreak
%    \nolinebreak
%    \newpage
%    \pagebreak
%    \nopagebreak

For page numbering, see Counters
To refer to a page number in the text, see \newline pageref
Return to LaTeX Table of Contents

\section{Hello Anywhere} u can write tex\\ from anywhere
		\subsection{MikTex} MikTex on
Windows
			\subsubsection{CTex}CTex integrate MikTex
				\paragraph{Editor}TeXworks
					\subparagraph{2 windows}one for edit one for review
		\subsection{Emacs}Emacs on Linux
			\subsubsection{Emacs-LaTex}

	\section{math}The Newton's second law is F=ma. %\\The Newton's second law is $F=ma$.  %\\just slice current line, and continue format-type?

	\paragraph{in} The Newton's second law is $F=ma$. % $$F=ma$$

	The Newton's second law is \[F=ma\].  %\\  %\\ can use independently to influence current line and next line's format

	x

	y\\
	z

	\paragraph{ou}
	Greek Letters $\eta$ and $\mu$.
%empty line start new box format  %\\ in fact continue inbox format, just like line-break...
	Fraction $\frac{a}{b}$

	Power $a^b$

	Subscript $a_b$\\
	Derivate $\frac{\partial y}{\partial t}$\\
	Vector $\vec{n}$\\
	Bold $\mathbf{n}$\\
	To time differential $\dot{F}$\\
	Matrix (lcr here means left , center or right for each column)
	\[
		\left[
			\begin{array}{llcr}
				a1 & b22 & c333 & mrrr\\
				d444 & e55555 & f6 & Po\\
%				I111  %llcr columns, just so so. llcr 4cols, l 1col...
			\end{array}
		\right]
	\]

	Equations(here \& is the symbol for aligning different rows)
	\begin{align}
		a+b&=c\\
		d=e+f&+g
	\end{align}

	\[
		\left\{ %note \{   in  \[
			\begin{aligned}
				&a+b=c\\
				&d=e+f+g\\
				&h=i+j
			\end{aligned} %aligned diff align
		%\right\}%right. for what?
			\begin{aligned}
				&I \\
				&II\\
				&III
			\end{aligned}
		\right\}%right. for what?
	\]
	
	\begin{center}
%	\includegraphics[width=7.00in,height=4.00in]{1,eps.jpg}
	\end{center}


	\begin{tabular}{ll|}
		%\cline{1-2}
		\hline
		\vline a1 &  \vline b21 \\
		c321 & d54321 \\
	\end{tabular}
	\begin{tabular}{|c|c|}
		\hline
		a & b \\
		\hline
		c & d \\
		\hline
	\end{tabular}	
	\begin{center}
		\begin{tabular}{|c|c|}
			\hline
			a & b \\ \hline
			c & d \\
			\hline
		\end{tabular}
	\end{center}

    \begin{tabular}{|c|c|c|}
      \hline
      % after \\: \hline or \cline{col1-col2} \cline{col3-col4} ...
      1 & 0 & 0 \\ \hline
      0 & 2 & 3 \\ \hline
      0 & 1 & 1 \\
      \hline
    \end{tabular}
	
	\begin{tabular}{|r|}
	•Q
	\end{tabular}



\end{document}
ignore me
忽略本初 ;)

















