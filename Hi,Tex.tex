\documentclass[a4paper,11pt]{book} %book,article,report,letter book(even,odd)...cool convenience
\usepackage{amsmath}
\usepackage{amssymb}
\usepackage{graphicx}
\usepackage{verbatim}
\author{William Song}
\title{Tex from Anywhere}

\begin{document}
%\maketitle title author datetime

\tableofcontents

\mainmatter
\part{chaos}
\chapter[short]{it's short really}
	\section{Hello Anywhere} u can write tex\\ from anywhere
		\subsection{MikTex} MikTex on
Windows
			\subsubsection{CTex}CTex integrate MikTex
				\paragraph{Editor}TeXworks
					\subparagraph{2 windows}one for edit one for review
		\subsection{Emacs}Emacs on Linux
			\subsubsection{Emacs-LaTex}
\paragraph{x}
use a b c d e f g H I J K L m 0 1 2 3 4 5 6 7 8 9 10 a b c d e f  \linebreak[3] Emacs-LaTex inside
\\
use a b c d e f g H I J K L m 0 1 2 3 4 5 6 7 8 9 10 a b c d e f g H I J K L m \\
% only undet book/reporte mode ...
balabal \ldots{} %...
\LaTeX2e
next \\%\newline
time\\
\# \$ \% \^{O} \& \_ \{ \} \~{A}
%\\[5pt]
$\backslash$
%\\[-5pt]


Today is \today
%\begin{comment}
%so, so
%you think
%you can tell
%heaven from hell
%\end{comment}
%awlful... break pages


	\section{math}The Newton's second law is F=ma. %\\The Newton's second law is $F=ma$.  %\\just slice current line, and continue format-type?

	\paragraph{in} The Newton's second law is $F=ma$. % $$F=ma$$

	The Newton's second law is \[F=ma\].  %\\  %\\ can use independently to influence current line and next line's format
	x\\
	y\\
	z

	\paragraph{ou}
	Greek Letters $\eta$ and $\mu$.
%empty line start new box format  %\\ in fact continue inbox format, just like line-break...
	Fraction $\frac{a}{b}$

	Power $a^b$

	Subscript $a_b$\\
	Derivate $\frac{\partial y}{\partial t}$\\
	Vector $\vec{n}$\\
	Bold $\mathbf{n}$\\
	To time differential $\dot{F}$\\
	Matrix (lcr here means left , center or right for each column)
	\[
		\left[
			\begin{array}{llcr}
				a1 & b22 & c333 & mrrr\\
				d444 & e55555 & f6 & Po\\
%				I111  %llcr columns, just so so. llcr 4cols, l 1col...
			\end{array}
		\right]
	\]

	Equations(here \& is the symbol for aligning different rows)
	\begin{align}
		a+b&=c\\
		d=e+f&+g
	\end{align}

	\[
		\left\{ %note \{   in  \[
			\begin{aligned}
				&a+b=c\\
				&d=e+f+g\\
				&h=i+j
			\end{aligned} %aligned diff align
		%\right\}%right. for what?
			\begin{aligned}
				&I \\
				&II\\
				&III
			\end{aligned}
		\right\}%right. for what?
	\]
	
	\begin{center}
%	\includegraphics[width=7.00in,height=4.00in]{1,eps.jpg}
	\end{center}


	\begin{tabular}{ll|}
		%\cline{1-2}
		\hline
		\vline a1 &  \vline b21 \\
		c321 & d54321 \\
	\end{tabular}
	\begin{tabular}{|c|c|}
		\hline
		a & b \\
		\hline
		c & d \\
		\hline
	\end{tabular}	
	\begin{center}
		\begin{tabular}{|c|c|}
			\hline
			a & b \\ \hline
			c & d \\
			\hline
		\end{tabular}
	\end{center}

    \begin{tabular}{|c|c|c|}
      \hline
      % after \\: \hline or \cline{col1-col2} \cline{col3-col4} ...
      1 & 0 & 0 \\ \hline
      0 & 2 & 3 \\ \hline
      0 & 1 & 1 \\
      \hline
    \end{tabular}
	
	\begin{tabular}{|r|}
	•Q
	\end{tabular}

\section{nasa-giss}
Hypertext Help with LaTeX
Line and Page Breaking


The first thing LaTeX does when processing ordinary text is to translate your input file into a string of glyphs and spaces. To produce a printed document, this string must be broken into lines, and these lines must be broken into pages. In some environments, you do the line breaking yourself with the \\ command, but LaTeX usually does it for you.
Commands for line and page breaks

%    \hyphenation
%
%    \cleardoublepage
%    \clearpage
%
%    \newline
%    \linebreak
%    \nolinebreak
%    \newpage
%    \pagebreak
%    \nopagebreak

For page numbering, see Counters
To refer to a page number in the text, see \\pageref
Return to LaTeX Table of Contents 

\end{document}
















